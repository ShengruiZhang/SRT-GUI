\documentclass{article}

\usepackage{amssymb}
\usepackage{graphicx}
\graphicspath{ {images/} }

\usepackage[a4paper, left=25mm, top=25mm]{geometry}
\usepackage[utf8]{inputenc}
\usepackage{vhistory}

\setlength{\parskip}{1em}
\setlength{\parindent}{0em}
\setlength{\baselineskip}{2em}


% --------------Main document------------------------------
% ---------------------------------------------------------
\begin{document}

% -------Title page begins-------------
\begin{titlepage}
  \begin{center}

    \vspace*{5cm}
    \huge Student Radio Telescope (SRT)

    \vspace*{2cm}
    \Large Graphical User Interface User Manual

    \vspace*{2cm}
    \large Document Number: 3000\\
    Revision: draft-0.2

    \vspace*{2.0cm}
    \large
    Team 21039\\
    Spring 2021

  \end{center}
\end{titlepage}
% -------Title page ends---------------

\tableofcontents
\pagebreak

\section{Revision History}
\begin{itemize}
  \item 2021-04-28: Draft created
\end{itemize}


\section{Graphical User Interface (GUI) Overview}
The Student Radio Telescope (SRT) Graphical User Interface (GUI) is the front-end for interfacing and controlling the SRT.

As of writing this manual, the GUI only have capability to perform basic telescope movement. Tracking and other improvement are future goals and were outside the requirement/scope of the Team 21039.

As of most other Capstone projects, this GUI is written in \textbf{Python} for simplicity and fast deployment. The GUI toolkit is \textbf{PySimpleGUI}, probably the easiest one to get started comparing to others.

Python3.7 was used during the development, newer version might be used but they are not tested and verified.

The GUI was developed under a Debian 10 GNU/Linux environment, so the rest of this manual assumes such. Since Python and the GUI toolkit are corss-platform, there should be no problem to fork the software to other platforms; only slight modification are needed such as the name of the USB-Serial port.

\section{Dependency}

The GUI is mainly written in Python, and Python3 and newer is required. To install python3 on Debian 10:\\
sudo apt install python3.7\\

To install other dependencies, Python's pip is recommended. To install pip on Debian 10:\\
sudo apt install python3-pip\\

With pip installed, check if package \textbf{PySimpleGUI} and \textbf{pyserial} exists using:\\
pip3 list\\

If not, install them using:\\
pip3 install PySimpleGUI\\
pip3 install pyserial\\

As of writing this manual, PySimpleGUI is on 4.30.0, and pyserial sits on 3.5

In an GNU/Linux environment, and after cloning the repo, the GUI can be opened via command line: python3.7 main.py

\section{GUI Layout}

If the GUI is opened correctly, there are should be four windows, the main control, two position dials, and a output window.

The main control is where essential operations take place; position dials simply keeps track of the position of the telescope in a visual way. The output window shows (error) message generated by the software for debugging.

Despite the output window mostly provides informational messages, it does output error and debug messages. Keep in mind that currently the outputs will not be saved upon exiting the GUI.

\section{Main Control Function}
This section documents the various functions of the main control GUI window. This window has the title: \textbf{\textit{Student Radio Telescope Control}}
\subsection{Menu Bar}

The menu bar contains three buttons: File, Edit, and Help.

From the file button you can exit the GUI.

The Edit button has the subbuttons of Edit, Copy, and Paste, which you can use when entering in coordinates.

The Help button will will show a link to this instruction manual.

\subsection{Three main buttons}

The three main buttons at the top of the GUI are Start Calibration, Telescope Stop, and Stow Telescope.

\subsubsection{Start Calibration}

The Start Calibration button should be pushed after the enable telescope and enable servomotor options are selected. This button runs a function which moves the telescope from its stow position to zenith.

\subsubsection{Telescope Stop}

The Telescope Stop button is available for users if a wrong input was pushed, or if there is a need for the telescope to be stopped immediately. This does not take the place of the emergency stop button. If a hard stop is needed, the user should use the red button that is immediately connected to the power supply. However, in the case of an incorrect entry, or any other reason, one can user the Telescope Stop button to stop the telescope immediately without shutting off the power or the GUI.

\subsubsection{Stow Telescope}

Once the user has finished using the PORT, they must stow the telescope by clicking this button before shutting down the GUI. This button takes in the current position of the telescope and moves the telescope to the stow position, 30 degrees past zenith. 

\subsection{Coordinate Entry}

The coordinate entry inputs are in terms of ALT/AZ. Therefore, the user must input only azimuthal and altitudinal values. The user simply has to enter a value between 15-120 degrees for the altitude, and a value between 0-360 for the azimuth axis. Then, you click the read button. The telescope then moves to the inputted coordinates. If the user enters a value that exceeds the mechanical limit, there is a software limit which will not allow the telescope to move past either 120 degrees altitudinally (stow position), or past horizon.
\subsection{System Status}
\subsection{Jogging}
\subsection{System Settings}
\subsection{Data Output}

\section{Operating Procedure}

\end{document}
